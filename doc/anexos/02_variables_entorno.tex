\chapter{Variables de entorno}
\label{sec:variables}

\textbf{IPManager} necesita algunas variables de entorno para funcionar correctamente. Todas tienen un valor por defecto, aunque se recomienda que se revisen y se modifiquen antes de comenzar a usar tanto backend como frontend.

\section{Backend}

Estas variables se pueden configurar como variables de entorno o se pueden configurar directamente en \textsf{config/server\_environment.py}.

\begin{itemize}
	\item \textsf{MONGO\_HOSTNAME}. Hostname donde se encuentra el host de \textit{MongoDB}. Por defecto \textsf{127.0.0.1}.
	\item \textsf{MONGO\_PORT}. Puerto del host de \textit{MongoDB}. Por defecto \textsf{27017}.
	\item \textsf{TESTING\_DATABASE}. Nombre de la colección usada para ejecutar los tests unitarios. Por defecto \textsf{ipm\_root\_testing}.
	\item \textsf{BASE\_DATABASE}. Nombre de la colección base de \textbf{IPManager}. En esta colección se almacenan datos de los clientes del backend. Por defecto toma el valor que tenga la variable \textsf{TESTING\_COLLECTION}.
	\item \textsf{ENC\_KEY}. Clave usada para encriptar las contraseñas. Puedes generar una clave ejecutando el archivo \textsf{generate\_key.py} que se encuentra en \textsf{utils}. Por defecto se usa una aleatoria.
	\item \textsf{JWT\_ENC\_KEY}. Clave usada para encriptar los \textit{JWT} usados en el login. Puedes generar una clave ejecutando el archivo \textsf{generate\_key.py} que se encuentra en el directorio \textit{src/utils}. Por defecto se usa una aleatoria.
	\item \textsf{DOCKER\_BASE\_URL}. \textit{URL} o \textit{path} donde se encuentra el socket de \textit{Docker}. Por defecto \textsf{unix://var/run/docker.sock}.
	\item \textsf{DOCKER\_ENABLED}. Activa o desactiva los endpoints para gestión de servicios. Por defecto comprueba si el backend se está ejecutando en un \textit{Docker} para desactivarlos en caso afirmativo.
	\item \textsf{ANSIBLE\_PATH}. \textit{Path} relativo o absoluto donde se van a almacenar los diferentes archivos generados por el backend necesarios para el aprovisionamiento. Por defecto \textsf{./}.
\end{itemize}


\section{Frontend}

El frontend cuenta con dos variables de entorno que se tienen que configurar previamente, son:

\begin{itemize}
	\item \textsf{backendUrl}. \textit{URL} del backend sin protocolo.
	\item \textsf{httpsEnabled}. \textit{True} si el backend cuenta con \textit{HTTPS}, \textit{false} en caso contrario.
\end{itemize}

La configuración de estas variables se puede hacer de dos maneras:

\begin{itemize}
	\item En los archivos \textsf{environment.*.ts} que se encuentran en \textsf{fronend/src/environments}. Estos archivos una vez configurados son inmutables una vez se ha construido el frontend.

	\begin{itemize}
		\item \textsf{environment.on-premise.ts}. Entorno utilizado para construir la imagen de \textit{Docker}. Por defecto se utilizan los valores usados en el \textit{docker-compose}.
		\item \textsf{environment.prod.ts}. Entorno utilizado para construir una versión de producción.
	\end{itemize}

	\item En el archivo \textsf{env.js}. Este archivo se encuentra en \textsf{fronend/src} y se puede modificar sin tener que reconstruir el frontend. En caso de no necesitar esta característica se recomienda utilizar el environment anterior. El archivo tiene la siguiente forma y se deben descomentar las líneas 2 a 5 para su uso.
	
\begin{lstlisting}
(function (window) {
// window.__env = {
//   backendUrl: '172.20.0.3:5000',
//   httpsEnabled: false
// };
}(this));
\end{lstlisting}
\end{itemize}
