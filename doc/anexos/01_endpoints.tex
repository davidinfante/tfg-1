\chapter{Especificación de \textit{endpoints}}

\textbf{Leyenda:}

\textbf{E/S}: Parámetro de \textbf{E}ntrada o \textbf{S}alida.


\section{Status}

\subsection{\textit{GET /status}}
Devuelve el estado actual de los dos servicios principales del backend, \textit{MongoDB} y \textit{Docker}.

Cabeceras necesarias:
\begin{table}[h!]
	\centering
	\begin{tabular}{|l|l|l|}
		\hline
		Nombre & Opcional & Descripción \\ \hline
		x-access-token & No & Token de acceso. \\ \hline
	\end{tabular}
\end{table}

Respuesta:
\begin{table}[h!]
	\centering
	\begin{adjustbox}{max width=\textwidth}
	\begin{tabular}{|l|l|l|l|}
		\hline
		Parámetro & Key & Tipo & Descripción \\ \hline
		mongo &  & dict &  \\ \hline
		& is\_up & bool & Si el servicio se encuentra activo o no. \\ \hline
		& data\_usage & list[dict] & Información de uso de datos de cada base de datos. \\ \hline
		& info & string & Información adicional del cliente de MongoDB. \\ \hline
		docker &  & dict &  \\ \hline
		& is\_up & bool & Si el servicio se encuentra activo o no. \\ \hline
		& data\_usage & dict & Información de uso de las imágenes y contenedores. \\ \hline
		& info & string & Información adicional del cliente de Docker. \\ \hline
	\end{tabular}
	\end{adjustbox}
\end{table}

\pagebreak
En caso de que el servicio de \textit{Docker} no se encuentre activado o presenta errores los \textit{endpoints} correspondientes al servicio de despliegue no estarán disponibles y el estado de este en la respuesta anterior tendrá la siguiente forma:

\begin{table}[h!]
	\centering
	
	\begin{adjustbox}{max width=\textwidth}
	\begin{tabular}{|l|l|l|l|}
		\hline
		Parámetro & Key & Tipo & Descripción \\ \hline
		docker &  & dict &  \\ \hline
		& status & bool & False. El servicio no funcina correctamente. \\ \hline
		& disabled & bool & Si el servicio se encuentra desactivado o no. \\ \hline
		& msg & string & Información adicional. \\ \hline
	\end{tabular}
	\end{adjustbox}
\end{table}






\section{Heartbeat}

\subsection{\textit{GET /api/heartbeat}}

Devuelve el estado actual de los dos servicios principales del backend de forma simplificada.

Respuesta:
\begin{table}[h!]
	\centering
\begin{adjustbox}{max width=\textwidth}
	\begin{tabular}{|l|l|l|}
		\hline
		Parámetro & Tipo & Descripción \\ \hline
		ok & bool & Si los servicios del sistema funcionan correctamente o no. \\ \hline
	\end{tabular}
\end{adjustbox}
\end{table}










\section{Autenticación}

\subsection{\textit{GET /login}}
Loguea al usuario en el cliente.

Cabeceras necesarias:
\begin{table}[h!]
	\centering
	\begin{adjustbox}{max width=\textwidth}
	\begin{tabular}{|l|l|l|}
		\hline
		Nombre & Opcional & Descripción \\ \hline
		x-access-token & No & Token de acceso. \\ \hline
	\end{tabular}
\end{adjustbox}
\end{table}


Respuesta:
\begin{table}[h!]
	\centering
	\begin{adjustbox}{max width=\textwidth}
	\begin{tabular}{|l|l|l|}
		\hline
		Parámetro & Tipo & Descripción \\ \hline
		token & string & Token JWT utilizado para identificar al usuario. \\ \hline
	\end{tabular}
\end{adjustbox}
\end{table}


\pagebreak
\subsection{\textit{POST /logout}}

Desloguea al usuario del cliente.

\bigskip
Cabeceras necesarias:

\begin{table}[h!]
	\centering
\begin{adjustbox}{max width=\textwidth}
	\begin{tabular}{|l|l|l|}
		\hline
		Nombre & Opcional & Descripción \\ \hline
		x-access-token & No & Token de acceso. \\ \hline
	\end{tabular}
\end{adjustbox}
\end{table}


Respuesta:
\begin{table}[h!]
	\centering
	\begin{adjustbox}{max width=\textwidth}
	\begin{tabular}{|l|l|l|}
		\hline
		Parámetro & Tipo & Descripción \\ \hline
		ok & bool & Si la operación se ha ejecutado correctamente o no. \\ \hline
		message & string & Mensaje complementario al estado de la operación. \\ \hline
	\end{tabular}
\end{adjustbox}
\end{table}




\section{Clientes}

\subsection{Cliente}
\label{sec:cliente}
\begin{table}[h!]
	\centering
	\begin{adjustbox}{max width=\textwidth}
	\begin{tabular}{|l|l|l|l|l|}
		\hline
		Parámetro & Tipo & Opcional & E/S & Descripción \\ \hline
		domain & string & No & E/S & Subdominio del cliente. \\ \hline
		db\_name & string & No & E/S & Base de datos del cliente. \\ \hline
	\end{tabular}
\end{adjustbox}
\end{table}


\subsection{\textit{POST /customer/query}}
Devuelve los clientes que cumplen los criterios de búsqueda.

Cabeceras necesarias:
\begin{table}[h!]
	\centering
	\begin{adjustbox}{max width=\textwidth}
	\begin{tabular}{|l|l|l|}
		\hline
		Nombre & Opcional & Descripción \\ \hline
		x-access-token & No & Token de acceso. \\ \hline
	\end{tabular}
\end{adjustbox}
\end{table}

Cuerpo de la petición (\textit{JSON}):
\begin{table}[h!]
	\centering
	\begin{adjustbox}{max width=\textwidth}
	\begin{tabular}{|l|l|l|l|}
		\hline
		Parámetro & Tipo & Opcional & Descripción \\ \hline
		query & dict & No & Criterio de búsqueda. \\ \hline
		filter & dict & Sí & Parámetros que se quieren en la respuesta. \\ \hline
	\end{tabular}
\end{adjustbox}
\end{table}

\pagebreak
Respuesta (un solo usuario):
\begin{table}[h!]
	\centering
	
	\begin{adjustbox}{max width=\textwidth}
	\begin{tabular}{|l|l|l|}
		\hline
		Parámetro & Tipo & Descripción \\ \hline
		data & \hyperref[sec:cliente]{Cliente} & Cliente que cumple el criterio de búsqueda. \\ \hline
	\end{tabular}
\end{adjustbox}
\end{table}

Respuesta (más de un usuario):
\begin{table}[h!]
	\centering
	\begin{adjustbox}{max width=\textwidth}
	\begin{tabular}{|l|l|l|}
		\hline
		Parámetro & Tipo & Descripción \\ \hline
		total & int & Número de clientes que cumplen el criterio de búsqueda. \\ \hline
		items & list[\hyperref[sec:cliente]{Cliente}] & Clientes que cumplen el criterio de búsqueda. \\ \hline
	\end{tabular}
\end{adjustbox}
\end{table}



\subsection{\textit{POST /customer}}
Crea un nuevo cliente.

Cabeceras necesarias:
\begin{table}[h!]
	\centering
	\begin{tabular}{|l|l|l|}
		\hline
		Nombre & Opcional & Descripción \\ \hline
		x-access-token & No & Token de acceso. \\ \hline
	\end{tabular}
\end{table}

Cuerpo de la petición (\textit{JSON}): \hyperref[sec:cliente]{Cliente}


Respuesta:
\begin{table}[h!]
	\centering
	\begin{adjustbox}{max width=\textwidth}
	\begin{tabular}{|l|l|l|}
		\hline
		Parámetro & Tipo & Descripción \\ \hline
		ok & bool & Si la operación se ha ejecutado correctamente o no. \\ \hline
		message & string & Mensaje complementario al estado de la operación. \\ \hline
	\end{tabular}
\end{adjustbox}
\end{table}



\subsection{\textit{PUT /customer}}
Modifica los datos de un cliente.

Cabeceras necesarias:
\begin{table}[h!]
	\centering
	\begin{adjustbox}{max width=\textwidth}
	\begin{tabular}{|l|l|l|}
		\hline
		Nombre & Opcional & Descripción \\ \hline
		x-access-token & No & Token de acceso. \\ \hline
	\end{tabular}
\end{adjustbox}
\end{table}

Cuerpo de la petición (\textit{JSON}):
\begin{table}[h!]
	\centering
	\begin{adjustbox}{max width=\textwidth}
	\begin{tabular}{|l|l|l|l|}
		\hline
		Parámetro & Tipo & Opcional & Descripción \\ \hline
		domain & string & No & Subominio del cliente que se quiere modificar. \\ \hline
		data & \hyperref[sec:cliente]{Cliente} & No & Nuevos datos del cliente. \\ \hline
	\end{tabular}
\end{adjustbox}
\end{table}

\pagebreak
Respuesta:
\begin{table}[h!]
	\centering
	\begin{adjustbox}{max width=\textwidth}
	\begin{tabular}{|l|l|l|}
		\hline
		Parámetro & Tipo & Descripción \\ \hline
		ok & bool & Si la operación se ha ejecutado correctamente o no. \\ \hline
		message & string & Mensaje complementario al estado de la operación. \\ \hline
	\end{tabular}
\end{adjustbox}
\end{table}






\subsection{\textit{DELETE /customer}}
Elimina un cliente.

Cabeceras necesarias:
\begin{table}[h!]
	\centering
	\begin{adjustbox}{max width=\textwidth}
	\begin{tabular}{|l|l|l|}
		\hline
		Nombre & Opcional & Descripción \\ \hline
		x-access-token & No & Token de acceso. \\ \hline
	\end{tabular}
\end{adjustbox}
\end{table}

Cuerpo de la petición (\textit{JSON}):
\begin{table}[h!]
	\centering
	\begin{adjustbox}{max width=\textwidth}
	\begin{tabular}{|l|l|l|l|}
		\hline
		Parámetro & Tipo & Opcional & Descripción \\ \hline
		domain & string & No & Subdominio del cliente que se quiere eliminar. \\ \hline
	\end{tabular}
\end{adjustbox}
\end{table}

Respuesta:
\begin{table}[h!]
	\centering
	\begin{adjustbox}{max width=\textwidth}
	\begin{tabular}{|l|l|l|}
		\hline
		Parámetro & Tipo & Descripción \\ \hline
		ok & bool & Si la operación se ha ejecutado correctamente o no. \\ \hline
		message & string & Mensaje complementario al estado de la operación. \\ \hline
	\end{tabular}
\end{adjustbox}
\end{table}


\section{Usuarios}

\subsection{Usuario}
\label{sec:usuario}
\begin{table}[h!]
	\centering
	\begin{adjustbox}{max width=\textwidth}
	\begin{tabular}{|l|l|l|l|l|}
		\hline
		Parámetro & Tipo & Opcional & E/S & Descripción \\ \hline
		type & string & No & E/S & Tipo de usuario. \\ \hline
		public\_id & string & - & S & UUID del usuario. \\ \hline
		first\_name & string & No & E/S & Nombre del usuario. \\ \hline
		last\_name & string & No & E/S & Apellido del usuario. \\ \hline
		username & string & No & E/S & Nickname del usuario. \\ \hline
		email & string & No & E/S & Email del usuario. \\ \hline
		password & string & No & E & Contraseña del usuario. \\ \hline
	\end{tabular}
\end{adjustbox}
\end{table}

\subsection{\textit{GET /user/:username}}
Devuelve toda la información asociada a un usuario.

Cabeceras necesarias:
\begin{table}[h!]
	\centering
	\begin{adjustbox}{max width=\textwidth}
	\begin{tabular}{|l|l|l|}
		\hline
		Nombre & Opcional & Descripción \\ \hline
		x-access-token & No & Token de acceso. \\ \hline
	\end{tabular}
\end{adjustbox}
\end{table}

Parámetros de la URL:
\begin{table}[h!]
	\centering
	\begin{adjustbox}{max width=\textwidth}
	\begin{tabular}{|l|l|l|}
		\hline
		Nombre & Opcional & Descripción \\ \hline
		username & No & Nombre del usuario a consultar. \\ \hline
	\end{tabular}
\end{adjustbox}
\end{table}

Respuesta:
\begin{table}[h!]
	\centering
	\begin{adjustbox}{max width=\textwidth}
	\begin{tabular}{|l|l|l|}
		\hline
		Parámetro & Tipo & Descripción \\ \hline
		data & \hyperref[sec:usuario]{Usuario} & Diccionario con toda la información del usuario consultado. \\ \hline
	\end{tabular}
\end{adjustbox}
\end{table}



\subsection{\textit{POST /user/query}}
Devuelve los usuarios que cumplen los criterios de búsqueda.

Cabeceras necesarias:
\begin{table}[h!]
	\centering
	\begin{adjustbox}{max width=\textwidth}
	\begin{tabular}{|l|l|l|}
		\hline
		Nombre & Opcional & Descripción \\ \hline
		x-access-token & No & Token de acceso. \\ \hline
	\end{tabular}
\end{adjustbox}
\end{table}


Cuerpo de la petición (\textit{JSON}):
\begin{table}[h!]
	\centering
	\begin{adjustbox}{max width=\textwidth}
	\begin{tabular}{|l|l|l|l|}
		\hline
		Parámetro & Tipo & Opcional & Descripción \\ \hline
		query & dict & No & Criterio de búsqueda. \\ \hline
		filter & dict & Sí & Parámetros que se quieren en la respuesta. \\ \hline
	\end{tabular}
\end{adjustbox}
\end{table}

Respuesta (un solo usuario):
\begin{table}[h!]
	\centering
	\begin{adjustbox}{max width=\textwidth}
	\begin{tabular}{|l|l|l|}
		\hline
		Parámetro & Tipo & Descripción \\ \hline
		data & \hyperref[sec:usuario]{Usuario} & Usuario que cumple el criterio de búsqueda. \\ \hline
	\end{tabular}
\end{adjustbox}
\end{table}

\pagebreak
Respuesta (más de un usuario):
\begin{table}[h!]
	\centering
	\begin{adjustbox}{max width=\textwidth}
	\begin{tabular}{|l|l|l|}
		\hline
		Parámetro & Tipo & Descripción \\ \hline
		total & int & Número de usuarios que cumplen el criterio de búsqueda. \\ \hline
		items & list[\hyperref[sec:usuario]{Usuario}] & Usuarios que cumplen el criterio de búsqueda. \\ \hline
	\end{tabular}
\end{adjustbox}
\end{table}

\subsection{\textit{POST /user}}
Crea un nuevo usuario.

Cabeceras necesarias:
\begin{table}[h!]
	\centering
	\begin{adjustbox}{max width=\textwidth}
	\begin{tabular}{|l|l|l|}
		\hline
		Nombre & Opcional & Descripción \\ \hline
		x-access-token & No & Token de acceso. \\ \hline
	\end{tabular}
\end{adjustbox}
\end{table}

Cuerpo de la petición (\textit{JSON}): \hyperref[sec:usuario]{Usuario}

Respuesta:
\begin{table}[h!]
	\centering
	\begin{adjustbox}{max width=\textwidth}
	\begin{tabular}{|l|l|l|}
		\hline
		Parámetro & Tipo & Descripción \\ \hline
		ok & bool & Si la operación se ha ejecutado correctamente o no. \\ \hline
		message & string & Mensaje complementario al estado de la operación. \\ \hline
	\end{tabular}
\end{adjustbox}
\end{table}




\subsection{\textit{PUT /user}}
Modifica los datos de un usuario.

Cabeceras necesarias:
\begin{table}[h!]
	\centering
	\begin{adjustbox}{max width=\textwidth}
	\begin{tabular}{|l|l|l|}
		\hline
		Nombre & Opcional & Descripción \\ \hline
		x-access-token & No & Token de acceso. \\ \hline
	\end{tabular}
\end{adjustbox}
\end{table}


Cuerpo de la petición (\textit{JSON}):
\begin{table}[h!]
	\centering
	\begin{adjustbox}{max width=\textwidth}
	\begin{tabular}{|l|l|l|l|}
		\hline
		Parámetro & Tipo & Opcional & Descripción \\ \hline
		email & string & No & Email del usuario que se quiere modificar. \\ \hline
		data & \hyperref[sec:usuario]{Usuario} & No & Nuevos datos del usuario. \\ \hline
	\end{tabular}
\end{adjustbox}
\end{table}

Respuesta:
\begin{table}[h!]
	\centering
	\begin{adjustbox}{max width=\textwidth}
	\begin{tabular}{|l|l|l|}
		\hline
		Parámetro & Tipo & Descripción \\ \hline
		ok & bool & Si la operación se ha ejecutado correctamente o no. \\ \hline
		message & string & Mensaje complementario al estado de la operación. \\ \hline
	\end{tabular}
\end{adjustbox}
\end{table}








\subsection{\textit{DELETE /user}}
Elimina un usuario.

Cabeceras necesarias:
\begin{table}[h!]
	\centering
	\begin{adjustbox}{max width=\textwidth}
	\begin{tabular}{|l|l|l|}
		\hline
		Nombre & Opcional & Descripción \\ \hline
		x-access-token & No & Token de acceso. \\ \hline
	\end{tabular}
\end{adjustbox}
\end{table}

Cuerpo de la petición (\textit{JSON}):
\begin{table}[h!]
	\centering
	\begin{adjustbox}{max width=\textwidth}
	\begin{tabular}{|l|l|l|l|}
		\hline
		Parámetro & Tipo & Opcional & Descripción \\ \hline
		email & string & No & Email del usuario que se quiere eliminar. \\ \hline
	\end{tabular}
\end{adjustbox}
\end{table}

Respuesta:
\begin{table}[h!]
	\centering
	\begin{adjustbox}{max width=\textwidth}
	\begin{tabular}{|l|l|l|}
		\hline
		Parámetro & Tipo & Descripción \\ \hline
		ok & bool & Si la operación se ha ejecutado correctamente o no. \\ \hline
		message & string & Mensaje complementario al estado de la operación. \\ \hline
	\end{tabular}
\end{adjustbox}
\end{table}





\section{Máquinas}

\subsection{Machine}
\label{sec:maquina}
\begin{table}[h!]
	\centering
	\begin{adjustbox}{max width=\textwidth}
	\begin{tabular}{|l|l|l|l|l|}
		\hline
		Parámetro & Tipo & Opcional & E/S & Descripción \\ \hline
		name & string & No & E/S & Tipo de usuario. \\ \hline
		description & string & Sí & E/S & Descripción de la máquina. \\ \hline
		type & string & No & E/S & Tipo de la máquina. \\ \hline
		ipv4 & string & Sí & E/S & Dirección IPv4 de la máquina. \\ \hline
		ipv6 & string & Sí & E/S & Dirección IPv6 de la máquina. \\ \hline
		mac & string & Sí & E/S & Dirección MAC de la máquina. \\ \hline
		broadcast & string & Sí & E/S & Broadcast de la red a la que se conecta la máquina. \\ \hline
		gateway & string & Sí & E/S & Gateway de la red a la que se conecta la máquina. \\ \hline
		netmask & string & Sí & E/S & Netmask de la red a la que se conecta la máquina. \\ \hline
		network & string & Sí & E/S & Network de la red a la que se conecta la máquina. \\ \hline
	\end{tabular}
\end{adjustbox}
\end{table}

\pagebreak
\subsection{\textit{GET /machine/:name}}
Devuelve toda la información asociada a una máquina.

Cabeceras necesarias:
\begin{table}[h!]
	\centering
	\begin{adjustbox}{max width=\textwidth}
	\begin{tabular}{|l|l|l|}
		\hline
		Nombre & Opcional & Descripción \\ \hline
		x-access-token & No & Token de acceso. \\ \hline
	\end{tabular}
\end{adjustbox}
\end{table}

Parámetros de la URL:
\begin{table}[h!]
	\centering
	\begin{adjustbox}{max width=\textwidth}
	\begin{tabular}{|l|l|l|}
		\hline
		Nombre & Opcional & Descripción \\ \hline
		name & No & Nombre de la máquina a consultar. \\ \hline
	\end{tabular}
\end{adjustbox}
\end{table}

Respuesta:
\begin{table}[h!]
	\centering
	\begin{adjustbox}{max width=\textwidth}
	\begin{tabular}{|l|l|l|}
		\hline
		Parámetro & Tipo & Descripción \\ \hline
		data & \hyperref[sec:maquina]{Machine} & Diccionario con toda la información de la máquina consultada. \\ \hline
	\end{tabular}
\end{adjustbox}
\end{table}



\subsection{\textit{POST/machine/query}}

Cabeceras necesarias:
\begin{table}[h!]
	\centering
	\begin{adjustbox}{max width=\textwidth}
	\begin{tabular}{|l|l|l|}
		\hline
		Nombre & Opcional & Descripción \\ \hline
		x-access-token & No & Token de acceso. \\ \hline
	\end{tabular}
\end{adjustbox}
\end{table}

Cuerpo de la petición (\textit{JSON}):
\begin{table}[h!]
	\centering
	\begin{adjustbox}{max width=\textwidth}
	\begin{tabular}{|l|l|l|l|}
		\hline
		Parámetro & Tipo & Opcional & Descripción \\ \hline
		query & dict & No & Criterio de búsqueda. \\ \hline
		filter & dict & Sí & Parámetros que se quieren en la respuesta. \\ \hline
	\end{tabular}
\end{adjustbox}
\end{table}

Respuesta (una sola máquina):
\begin{table}[h!]
	\centering
	\begin{adjustbox}{max width=\textwidth}
	\begin{tabular}{|l|l|l|}
		\hline
		Parámetro & Tipo & Descripción \\ \hline
		data & \hyperref[sec:maquina]{Machine} & Máquina que cumple el criterio de búsqueda. \\ \hline
	\end{tabular}
\end{adjustbox}
\end{table}

Respuesta (más de una máquina):
\begin{table}[h!]
	\centering
	\begin{adjustbox}{max width=\textwidth}
	\begin{tabular}{|l|l|l|}
		\hline
		Parámetro & Tipo & Descripción \\ \hline
		total & int & Número de máquinas que cumplen el criterio de búsqueda. \\ \hline
		items & list[\hyperref[sec:maquina]{Machine}] & Máquinas que cumplen el criterio de búsqueda. \\ \hline
	\end{tabular}
\end{adjustbox}
\end{table}



\subsection{\textit{POST /machine}}

Cabeceras necesarias:
\begin{table}[h!]
	\centering
	\begin{adjustbox}{max width=\textwidth}
	\begin{tabular}{|l|l|l|}
		\hline
		Nombre & Opcional & Descripción \\ \hline
		x-access-token & No & Token de acceso. \\ \hline
	\end{tabular}
\end{adjustbox}
\end{table}

Cuerpo de la petición (\textit{JSON}): \hyperref[sec:maquina]{Machine}

Respuesta:
\begin{table}[h!]
	\centering
	\begin{adjustbox}{max width=\textwidth}
	\begin{tabular}{|l|l|l|}
		\hline
		Parámetro & Tipo & Descripción \\ \hline
		ok & bool & Si la operación se ha ejecutado correctamente o no. \\ \hline
		message & string & Mensaje complementario al estado de la operación. \\ \hline
	\end{tabular}
\end{adjustbox}
\end{table}


\subsection{\textit{PUT /machine}}

Cabeceras necesarias:
\begin{table}[h!]
	\centering
	\begin{adjustbox}{max width=\textwidth}
	\begin{tabular}{|l|l|l|}
		\hline
		Nombre & Opcional & Descripción \\ \hline
		x-access-token & No & Token de acceso. \\ \hline
	\end{tabular}
\end{adjustbox}
\end{table}

Cuerpo de la petición (\textit{JSON}):
\begin{table}[h!]
	\centering
	\begin{adjustbox}{max width=\textwidth}
	\begin{tabular}{|l|l|l|l|}
		\hline
		Parámetro & Tipo & Opcional & Descripción \\ \hline
		name & string & No & Nombre de la máquina que se quiere modificar. \\ \hline
		data & \hyperref[sec:maquina]{Machine} & No & Nuevos datos de la máquina. \\ \hline
	\end{tabular}
\end{adjustbox}
\end{table}


Respuesta:
\begin{table}[h!]
	\centering
	\begin{adjustbox}{max width=\textwidth}
	\begin{tabular}{|l|l|l|}
		\hline
		Parámetro & Tipo & Descripción \\ \hline
		ok & bool & Si la operación se ha ejecutado correctamente o no. \\ \hline
		message & string & Mensaje complementario al estado de la operación. \\ \hline
	\end{tabular}
\end{adjustbox}
\end{table}


\subsection{\textit{DELETE /machine}}

Cabeceras necesarias:
\begin{table}[h!]
	\centering
	\begin{adjustbox}{max width=\textwidth}
	\begin{tabular}{|l|l|l|}
		\hline
		Nombre & Opcional & Descripción \\ \hline
		x-access-token & No & Token de acceso. \\ \hline
	\end{tabular}
\end{adjustbox}
\end{table}

\pagebreak

Cuerpo de la petición (\textit{JSON}):
\begin{table}[h!]
	\centering
	\begin{adjustbox}{max width=\textwidth}
	\begin{tabular}{|l|l|l|l|}
		\hline
		Parámetro & Tipo & Opcional & Descripción \\ \hline
		name & string & No & Nombre de la máquina que se quiere eliminar. \\ \hline
	\end{tabular}
\end{adjustbox}
\end{table}

Respuesta:
\begin{table}[h!]
	\centering
	\begin{adjustbox}{max width=\textwidth}
	\begin{tabular}{|l|l|l|}
		\hline
		Parámetro & Tipo & Descripción \\ \hline
		ok & bool & Si la operación se ha ejecutado correctamente o no. \\ \hline
		message & string & Mensaje complementario al estado de la operación. \\ \hline
	\end{tabular}
\end{adjustbox}
\end{table}










\section{Grupos de hosts}

\subsection{\textit{Hosts}}
\label{sec:hosts}
\begin{table}[h!]
	\centering
	\begin{adjustbox}{max width=\textwidth}
	\begin{tabular}{|l|l|l|l|l|}
		\hline
		Parámetro & Tipo & Opcional & E/S & Descripción \\ \hline
		name & string & No & E/S & Nombre del grupo de hosts. \\ \hline
		ips & list[string] & No & E/S & Direcciones IPv4. \\ \hline
	\end{tabular}
\end{adjustbox}
\end{table}

\subsection{\textit{GET /provision/hosts/:name}}

Devuelve toda la información asociada a un grupo de \textit{hosts}.

Cabeceras necesarias:

\begin{table}[h!]
	\centering
	\begin{adjustbox}{max width=\textwidth}
	\begin{tabular}{|l|l|l|}
		\hline
		Nombre & Opcional & Descripción \\ \hline
		x-access-token & No & Token de acceso. \\ \hline
	\end{tabular}
\end{adjustbox}
\end{table}

Parámetros de la URL:

\begin{table}[h!]
	\centering
	\begin{adjustbox}{max width=\textwidth}
	\begin{tabular}{|l|l|l|}
		\hline
		Nombre & Opcional & Descripción \\ \hline
		name & No & Nombre del grupo de hosts a consultar. \\ \hline
	\end{tabular}
\end{adjustbox}
\end{table}

Respuesta:

\begin{table}[h!]
	\centering
	\begin{adjustbox}{max width=\textwidth}
	\begin{tabular}{|l|l|l|}
		\hline
		Parámetro & Tipo & Descripción \\ \hline
		data & \hyperref[sec:hosts]{Hosts} & Diccionario con toda la información del grupo de hosts consultado. \\ \hline
	\end{tabular}
\end{adjustbox}
\end{table}


\pagebreak
\subsection{\textit{POST/provision/hosts/query}}

Devuelve los grupos de \textit{hosts} que cumplen los criterios de búsqueda.

Cabeceras necesarias:

\begin{table}[h!]
	\centering
	\begin{adjustbox}{max width=\textwidth}
	\begin{tabular}{|l|l|l|}
		\hline
		Nombre & Opcional & Descripción \\ \hline
		x-access-token & No & Token de acceso. \\ \hline
	\end{tabular}
\end{adjustbox}
\end{table}

Cuerpo de la petición (\textit{JSON}):
\begin{table}[h!]
	\centering
	\begin{adjustbox}{max width=\textwidth}
	\begin{tabular}{|l|l|l|l|}
		\hline
		Parámetro & Tipo & Opcional & Descripción \\ \hline
		query & dict & No & Criterio de búsqueda. \\ \hline
		filter & dict & Sí & Parámetros que se quieren en la respuesta. \\ \hline
	\end{tabular}
\end{adjustbox}
\end{table}

Respuesta (una sola máquina):
\begin{table}[h!]
	\centering
	\begin{adjustbox}{max width=\textwidth}
	\begin{tabular}{|l|l|l|}
		\hline
		Parámetro & Tipo & Descripción \\ \hline
		data & \hyperref[sec:hosts]{Hosts} & Grupo de hosts que cumple el criterio de búsqueda. \\ \hline
	\end{tabular}
\end{adjustbox}
\end{table}

Respuesta (más de una máquina):
\begin{table}[h!]
	\centering
	\begin{adjustbox}{max width=\textwidth}
	\begin{tabular}{|l|l|l|}
		\hline
		Parámetro & Tipo & Descripción \\ \hline
		total & int & Número de grupos de hosts que cumplen el criterio de búsqueda. \\ \hline
		items & list[\hyperref[sec:hosts]{Hosts}] & Grupos de hosts que cumplen el criterio de búsqueda. \\ \hline
	\end{tabular}
\end{adjustbox}
\end{table}



\subsection{\textit{POST /provision/hosts}}

Cabeceras necesarias:
\begin{table}[h!]
	\centering
	\begin{adjustbox}{max width=\textwidth}
	\begin{tabular}{|l|l|l|}
		\hline
		Nombre & Opcional & Descripción \\ \hline
		x-access-token & No & Token de acceso. \\ \hline
	\end{tabular}
\end{adjustbox}
\end{table}

Cuerpo de la petición (\textit{JSON}): \hyperref[sec:hosts]{Hosts}

Respuesta:
\begin{table}[h!]
	\centering
	\begin{adjustbox}{max width=\textwidth}
	\begin{tabular}{|l|l|l|}
		\hline
		Parámetro & Tipo & Descripción \\ \hline
		ok & bool & Si la operación se ha ejecutado correctamente o no. \\ \hline
		message & string & Mensaje complementario al estado de la operación. \\ \hline
	\end{tabular}
\end{adjustbox}
\end{table}

\pagebreak
\subsection{\textit{PUT /provision/hosts}}

Cabeceras necesarias:
\begin{table}[h!]
	\centering
	\begin{adjustbox}{max width=\textwidth}
	\begin{tabular}{|l|l|l|}
		\hline
		Nombre & Opcional & Descripción \\ \hline
		x-access-token & No & Token de acceso. \\ \hline
	\end{tabular}
\end{adjustbox}
\end{table}

Cuerpo de la petición (\textit{JSON}):
\begin{table}[h!]
	\centering
	\begin{adjustbox}{max width=\textwidth}
	\begin{tabular}{|l|l|l|l|}
		\hline
		Parámetro & Tipo & Opcional & Descripción \\ \hline
		name & string & No & Nombre del grupo de hosts que se quiere modificar. \\ \hline
		data & \hyperref[sec:hosts]{Hosts} & No & Nuevos datos del grupo de hosts. \\ \hline
	\end{tabular}
\end{adjustbox}
\end{table}


Respuesta:
\begin{table}[h!]
	\centering
	\begin{adjustbox}{max width=\textwidth}
	\begin{tabular}{|l|l|l|}
		\hline
		Parámetro & Tipo & Descripción \\ \hline
		ok & bool & Si la operación se ha ejecutado correctamente o no. \\ \hline
		message & string & Mensaje complementario al estado de la operación. \\ \hline
	\end{tabular}
\end{adjustbox}
\end{table}

\subsection{\textit{DELETE /provision/hosts}}

Cabeceras necesarias:
\begin{table}[h!]
	\centering
	\begin{adjustbox}{max width=\textwidth}
	\begin{tabular}{|l|l|l|}
		\hline
		Nombre & Opcional & Descripción \\ \hline
		x-access-token & No & Token de acceso. \\ \hline
	\end{tabular}
\end{adjustbox}
\end{table}

Cuerpo de la petición (\textit{JSON}):
\begin{table}[h!]
	\centering
	\begin{adjustbox}{max width=\textwidth}
	\begin{tabular}{|l|l|l|l|}
		\hline
		Parámetro & Tipo & Opcional & Descripción \\ \hline
		name & string & No & Nombre del grupo de hosts que se quiere eliminar. \\ \hline
	\end{tabular}
\end{adjustbox}
\end{table}

Respuesta:
\begin{table}[h!]
	\centering
	\begin{adjustbox}{max width=\textwidth}
	\begin{tabular}{|l|l|l|}
		\hline
		Parámetro & Tipo & Descripción \\ \hline
		ok & bool & Si la operación se ha ejecutado correctamente o no. \\ \hline
		message & string & Mensaje complementario al estado de la operación. \\ \hline
	\end{tabular}
\end{adjustbox}
\end{table}









\pagebreak
\section{Playbooks}

	\subsection{Playbook}
	\label{sec:playbook}
		\begin{table}[h!]
			\centering
	\begin{adjustbox}{max width=\textwidth}
			\begin{tabular}{|l|l|l|l|l|}
				\hline
				Parámetro & Tipo & Opcional & E/S & Descripción \\ \hline
				name & string & No & E/S & Nombre del Playbook. \\ \hline
				playbook & dict & No & E/S & Contenido del Playbook codificado como \textit{JSON}. \\ \hline
			\end{tabular}
\end{adjustbox}
		\end{table}
	
	\subsection{\textit{GET /provision/playbook/:name}}
		Devuelve toda la información asociada a un Playbook.
		
		Cabeceras necesarias:
		\begin{table}[h!]
			\centering
	\begin{adjustbox}{max width=\textwidth}
			\begin{tabular}{|l|l|l|}
				\hline
				Nombre & Opcional & Descripción \\ \hline
				x-access-token & No & Token de acceso. \\ \hline
			\end{tabular}
\end{adjustbox}
		\end{table}
		
		Parámetros de la URL:
		\begin{table}[h!]
			\centering
	\begin{adjustbox}{max width=\textwidth}
			\begin{tabular}{|l|l|l|}
				\hline
				Nombre & Opcional & Descripción \\ \hline
				name & No & Nombre del Playbook a consultar. \\ \hline
			\end{tabular}
\end{adjustbox}
		\end{table}
		
		Respuesta:
		\begin{table}[h!]
			\centering
	\begin{adjustbox}{max width=\textwidth}
			\begin{tabular}{|l|l|l|}
				\hline
				Parámetro & Tipo & Descripción \\ \hline
				data & \hyperref[sec:hosts]{Hosts} & Diccionario con toda la información del Playbook consultado. \\ \hline
			\end{tabular}
\end{adjustbox}
		\end{table}
	
	
	
	\subsection{\textit{POST/provision/playbook/query}}
		Devuelve los \textit{playbooks} que cumplen los criterios de búsqueda.
		
		Cabeceras necesarias:
		\begin{table}[h!]
			\centering
	\begin{adjustbox}{max width=\textwidth}
			\begin{tabular}{|l|l|l|}
				\hline
				Nombre & Opcional & Descripción \\ \hline
				x-access-token & No & Token de acceso. \\ \hline
			\end{tabular}
\end{adjustbox}
		\end{table}
		
		Cuerpo de la petición (\textit{JSON}):
		\begin{table}[h!]
			\centering
	\begin{adjustbox}{max width=\textwidth}
			\begin{tabular}{|l|l|l|l|}
				\hline
				Parámetro & Tipo & Opcional & Descripción \\ \hline
				query & dict & No & Criterio de búsqueda. \\ \hline
				filter & dict & Sí & Parámetros que se quieren en la respuesta. \\ \hline
			\end{tabular}
\end{adjustbox}
		\end{table}
		
		
		\pagebreak
		Respuesta (un solo \textit{Playbook}):
		\begin{table}[h!]
			\centering
	\begin{adjustbox}{max width=\textwidth}
			\begin{tabular}{|l|l|l|}
				\hline
				Parámetro & Tipo & Descripción \\ \hline
				data & \hyperref[sec:playbook]{Playbook} & Playbook que cumple el criterio de búsqueda. \\ \hline
			\end{tabular}
\end{adjustbox}
		\end{table}
		
		Respuesta (más de un \textit{Playbook}):
		\begin{table}[h!]
			\centering
	\begin{adjustbox}{max width=\textwidth}
			\begin{tabular}{|l|l|l|}
				\hline
				Parámetro & Tipo & Descripción \\ \hline
				total & int & Número de Playbooks que cumplen el criterio de búsqueda. \\ \hline
				items & list[\hyperref[sec:playbook]{Playbook}] & Playbooks que cumplen el criterio de búsqueda. \\ \hline
			\end{tabular}
\end{adjustbox}
		\end{table}
	
	
	
	\subsection{\textit{POST /provision/playbook}}
		Crea un nuevo \textit{Playbook}.
		
		Cabeceras necesarias:
		\begin{table}[h!]
			\centering
	\begin{adjustbox}{max width=\textwidth}
			\begin{tabular}{|l|l|l|}
				\hline
				Nombre & Opcional & Descripción \\ \hline
				x-access-token & No & Token de acceso. \\ \hline
			\end{tabular}
\end{adjustbox}
		\end{table}
		
		Cuerpo de la petición (\textit{JSON}): \hyperref[sec:playbook]{Playbook}
		
		Respuesta:
		\begin{table}[h!]
			\centering
	\begin{adjustbox}{max width=\textwidth}
			\begin{tabular}{|l|l|l|}
				\hline
				Parámetro & Tipo & Descripción \\ \hline
				ok & bool & Si la operación se ha ejecutado correctamente o no. \\ \hline
				message & string & Mensaje complementario al estado de la operación. \\ \hline
			\end{tabular}
\end{adjustbox}
	\end{table}
	
	
	\subsection{\textit{PUT /provision/playbook}}
		Modifica los datos de un \textit{Playbook}.
		
		Cabeceras necesarias:
		\begin{table}[h!]
			\centering
	\begin{adjustbox}{max width=\textwidth}
			\begin{tabular}{|l|l|l|}
				\hline
				Nombre & Opcional & Descripción \\ \hline
				x-access-token & No & Token de acceso. \\ \hline
			\end{tabular}
\end{adjustbox}
		\end{table}
		
		Cuerpo de la petición (\textit{JSON}):
		\begin{table}[h!]
			\centering
	\begin{adjustbox}{max width=\textwidth}
			\begin{tabular}{|l|l|l|l|}
				\hline
				Parámetro & Tipo & Opcional & Descripción \\ \hline
				name & string & No & Nombre del grupo de hosts que se quiere modificar. \\ \hline
				data & \hyperref[sec:playbook]{Playbook} & No & Nuevos datos del grupo de hosts. \\ \hline
			\end{tabular}
\end{adjustbox}
		\end{table}
		
		
		\pagebreak
		Respuesta:
		\begin{table}[h!]
			\centering
	\begin{adjustbox}{max width=\textwidth}
			\begin{tabular}{|l|l|l|}
				\hline
				Parámetro & Tipo & Descripción \\ \hline
				ok & bool & Si la operación se ha ejecutado correctamente o no. \\ \hline
				message & string & Mensaje complementario al estado de la operación. \\ \hline
			\end{tabular}
\end{adjustbox}
		\end{table}
	
	\subsection{\textit{DELETE /provision/playbook}}
		Elimina un \textit{Playbook}.
		
		Cabeceras necesarias:
		\begin{table}[h!]
			\centering
	\begin{adjustbox}{max width=\textwidth}
			\begin{tabular}{|l|l|l|}
				\hline
				Nombre & Opcional & Descripción \\ \hline
				x-access-token & No & Token de acceso. \\ \hline
			\end{tabular}
\end{adjustbox}
		\end{table}
		
		Cuerpo de la petición (\textit{JSON}):
		\begin{table}[h!]
			\centering
	\begin{adjustbox}{max width=\textwidth}
			\begin{tabular}{|l|l|l|l|}
				\hline
				Parámetro & Tipo & Opcional & Descripción \\ \hline
				name & string & No & Nombre del Playbook que se quiere eliminar. \\ \hline
			\end{tabular}
\end{adjustbox}
		\end{table}
		
		Respuesta:
		\begin{table}[h!]
			\centering
	\begin{adjustbox}{max width=\textwidth}
			\begin{tabular}{|l|l|l|}
				\hline
				Parámetro & Tipo & Descripción \\ \hline
				ok & bool & Si la operación se ha ejecutado correctamente o no. \\ \hline
				message & string & Mensaje complementario al estado de la operación. \\ \hline
			\end{tabular}
\end{adjustbox}
		\end{table}











\section{Aprovisionamiento}


	\subsection{\textit{POST /provision}}
		Ejecuta un \textit{Playbook}.
	
		Cabeceras necesarias:
		\begin{table}[h!]
			\centering
	\begin{adjustbox}{max width=\textwidth}
			\begin{tabular}{|l|l|l|}
				\hline
				Nombre & Opcional & Descripción \\ \hline
				x-access-token & No & Token de acceso. \\ \hline
			\end{tabular}
\end{adjustbox}
		\end{table}
		
		\pagebreak
		Cuerpo de la petición (\textit{JSON}):
		\begin{table}[h!]
			\centering
	\begin{adjustbox}{max width=\textwidth}
			\begin{tabular}{|l|l|l|l|l|}
				\hline
				Parámetro & Key & Tipo & Opcional & Descripción \\ \hline
				hosts &  & list[string] & No & Lista de grupo de hosts donde se quiere ejecutar el playbook. \\ \hline
				playbook &  & string & No & Nombre del playbook a ejecutar. \\ \hline
				passwords &  & dict & No & Contraseñas necesarias para la conexión a los hosts. \\ \hline
				& conn\_pass & string & Sí & Contraseña de acceso. \\ \hline
				& become\_pass & string & Sí & Contraseña para acceder al root. \\ \hline
			\end{tabular}
\end{adjustbox}
		\end{table}
		
		Respuesta:
		\begin{table}[h!]
			\centering
	\begin{adjustbox}{max width=\textwidth}
			\begin{tabular}{|l|l|l|}
				\hline
				Parámetro & Tipo & Descripción \\ \hline
				result & string & Respuesta de la ejecución del playbook. \\ \hline
			\end{tabular}
\end{adjustbox}
		\end{table}







\section{Despliegue}

	\subsection{Contenedor}
	\label{sec:contenedor}
		\begin{table}[h!]
			\centering
	\begin{adjustbox}{max width=\textwidth}
			\begin{tabular}{|l|l|l|l|}
				\hline
				Parámetro & Tipo & E/S & Descripción \\ \hline
				id & string & S & Identificador del contenedor \\ \hline
				short\_id & string & S & Identificador del contenedor truncado a 10 carácteres. \\ \hline
				name & string & S & Nombre del contenedor. \\ \hline
				labels & dict & S & Etiquetas del contenedor. \\ \hline
				status & string & S & Estado del contenedor. \\ \hline
				image & Image & S & Imagen que se esta ejecutando en el contenedor. \\ \hline
			\end{tabular}
\end{adjustbox}
		\end{table}
	
	\subsection{Imagen}
	\label{sec:imagen}
		\begin{table}[h!]
			\centering
	\begin{adjustbox}{max width=\textwidth}
			\begin{tabular}{|l|l|l|l|}
				\hline
				Parámetro & Tipo & E/S & Descripción \\ \hline
				id & string & S & Identificador de la imagen. \\ \hline
				labels & dict & S & Etiquetas de la imagen. \\ \hline
				short\_id & string & S & Identificador de la imagen truncado a 10 carácteres. \\ \hline
				tags & list[string] & S & Tags de la imagen. \\ \hline
			\end{tabular}
\end{adjustbox}
		\end{table}
	
	\subsection{Filtro de contenedores}
	\label{sec:filtrocontenedor}
		\begin{table}[h!]
			\centering
	\begin{adjustbox}{max width=\textwidth}
			\begin{tabular}{|l|l|l|l|l|}
				\hline
				Parámetro & Key & Tipo & Opcional & Descripción \\ \hline
				filters &  & dict & No &  \\ \hline
				& exited & boolean & Sí & Si el contenedor ha finalizado su ejecución o no. \\ \hline
				& status & string & Sí & Estado del contenedor. \\ \hline
				& id & string & Sí & Identificador del contenedor. \\ \hline
				& name & string & Sí & Nombre del contenedor. \\ \hline
			\end{tabular}
\end{adjustbox}
		\end{table}
	
	\subsection{Filtro de imágenes}
	\label{sec:filtroimagen}
		\begin{table}[h!]
			\centering
	\begin{adjustbox}{max width=\textwidth}
			\begin{tabular}{|l|l|l|l|l|}
				\hline
				Parámetro & Key & Tipo & Opcional & Descripción \\ \hline
				filters &  & dict & No &  \\ \hline
				& dangling & boolean & Sí & Si la imagen se encuentra colgada o no. \\ \hline
				& label & string & Sí & Etiqueta de la imagen. \\ \hline
			\end{tabular}
\end{adjustbox}
		\end{table}
	
	
	\subsection{\textit{POST /deploy/container}}
		Permite ejecutar operaciones básicas en todos los contenedores.
		
		Cabeceras necesarias:
		\begin{table}[h!]
			\centering
	\begin{adjustbox}{max width=\textwidth}
			\begin{tabular}{|l|l|l|}
				\hline
				Nombre & Opcional & Descripción \\ \hline
				x-access-token & No & Token de acceso. \\ \hline
			\end{tabular}
\end{adjustbox}
		\end{table}
		
		Cuerpo de la petición (\textit{JSON}):
		\begin{table}[h!]
			\centering
	\begin{adjustbox}{max width=\textwidth}
			\begin{tabular}{|l|l|l|l|}
				\hline
				Parámetro & Tipo & Opcional & Descripción \\ \hline
				operation & string & No & Nombre de la operación que se quiere ejecutar. Valores posibles: \textit{run}, \textit{get}, \textit{list}, \textit{prune}. \\ \hline
				data & dict & No & Argumentos de la operación. \\ \hline
			\end{tabular}
\end{adjustbox}
		\end{table}
	
		Tipos de operaciones:
		\subsubsection{run}
			\begin{table}[h!]
				\centering
	\begin{adjustbox}{max width=\textwidth}
				\begin{tabular}{|l|l|l|l|l|}
					\hline
					Parámetro & Key & Tipo & Opcional & Descripción \\ \hline
					data &  & dict & No & Argumentos de la operación. \\ \hline
					& image & string & No & Imagen a ejecutar. \\ \hline
					& command & list[string] & Sí & Comando a ejecutar en el contenedor. \\ \hline
					& auto\_remove & bool & Sí & Eliminar o no el contenedor al terminar la ejecución. \\ \hline
					& detach & bool & Sí & Ejecutar o no el contenedor en segundo plano. Por defecto \textit{true}. \\ \hline
					& entrypoint & list[string] & Sí & Entrypoint del contenedor. \\ \hline
					& environment & dict & Sí & Variables de entorno. \\ \hline
					& hostname & string & Sí & Hostname del contenedor. \\ \hline
					& mounts & list[string] & Sí & Lista de volumenes que se montan en el contenedor. \\ \hline
					& name & string & Sí & Nombre del contenedor. \\ \hline
					& network & string & Sí & Nombre de la red a la que se conecta. \\ \hline
					& ports & dict & Sí & Puertos a enlazar. \\ \hline
					& user & string & Sí & Usuario para ejecutar los posibles comandos dentro del contenedor. \\ \hline
					& volumes & dict & Sí & Volumenes a montar. \\ \hline
					& working\_dir & string & Sí & Directorio de trabajo. \\ \hline
					& remove & bool & Sí & Eliminar el contenedor al terminar la ejecución. \\ \hline
				\end{tabular}
\end{adjustbox}
			\end{table}
		
		\pagebreak
		\subsubsection{get}
			\begin{table}[h!]
				\centering
	\begin{adjustbox}{max width=\textwidth}
				\begin{tabular}{|l|l|l|l|l|}
					\hline
					Parámetro & Key & Tipo & Opcional & Descripción \\ \hline
					data &  & dict & No & Argumentos de la operación. \\ \hline
					& container\_id & string & No & Identificador del contenedor \\ \hline
				\end{tabular}
\end{adjustbox}
			\end{table}
		
			Respuesta: \hyperref[sec:contenedor]{Contenedor}
			
		\subsubsection{list}
			\begin{table}[h!]
				\centering
	\begin{adjustbox}{max width=\textwidth}
				\begin{tabular}{|l|l|l|l|l|}
					\hline
					Parámetro & Key & Tipo & Opcional & Descripción \\ \hline
					data &  & dict & No & Argumentos de la operación. \\ \hline
					& all & bool & No & Mostrar o no todos los contenedores (en ejecución y detenidos o finalizados). \\ \hline
					& since & string & No & Mostrar contenedores creados desde este identificador. \\ \hline
					& before & string & No & Mostrar contenedores creados previos a este identificador. \\ \hline
					& filters & \hyperref[sec:filtrocontenedor]{Filtro} & No & Filtros para afinar la búsqueda. \\ \hline
				\end{tabular}
\end{adjustbox}
			\end{table}
		
			Respuesta (un solo contenedor):
			\begin{table}[h!]
				\centering
	\begin{adjustbox}{max width=\textwidth}
				\begin{tabular}{|l|l|l|}
					\hline
					Parámetro & Tipo & Descripción \\ \hline
					data & \hyperref[sec:contenedor]{Contenedor} & Contenedor que cumple el criterio de búsqueda. \\ \hline
				\end{tabular}
\end{adjustbox}
			\end{table}
		
			Respuesta (más de un contenedor):
			\begin{table}[h!]
				\centering
	\begin{adjustbox}{max width=\textwidth}
				\begin{tabular}{|l|l|l|}
					\hline
					Parámetro & Tipo & Descripción \\ \hline
					total & int & Número de contenedores que cumplen el criterio de búsqueda. \\ \hline
					items & list[\hyperref[sec:contenedor]{Contenedor}] & Contenedores que cumplen el criterio de búsqueda. \\ \hline
				\end{tabular}
\end{adjustbox}
			\end{table}
		
		\subsubsection{prune}
			\begin{table}[h!]
				\centering
	\begin{adjustbox}{max width=\textwidth}
				\begin{tabular}{|l|l|l|l|l|}
					\hline
					Parámetro & Key & Tipo & Opcional & Descripción \\ \hline
					data &  & dict & No & Argumentos de la operación. \\ \hline
					& filters & \hyperref[sec:filtrocontenedor]{Filtro} & Sí & Filtros. \\ \hline
				\end{tabular}
\end{adjustbox}
			\end{table}
	
	
	

	\subsection{\textit{POST /deploy/container/single}}
		Permite ejecutar operaciones básicas en un contenedor en concreto.
	
		Cabeceras necesarias:
		\begin{table}[h!]
			\centering
	\begin{adjustbox}{max width=\textwidth}
			\begin{tabular}{|l|l|l|}
				\hline
				Nombre & Opcional & Descripción \\ \hline
				x-access-token & No & Token de acceso. \\ \hline
			\end{tabular}
\end{adjustbox}
		\end{table}
		
		\pagebreak
		Cuerpo de la petición (\textit{JSON}):
		\begin{table}[h!]
			\centering
	\begin{adjustbox}{max width=\textwidth}
			\begin{tabular}{|l|l|l|l|}
				\hline
				Parámetro & Tipo & Opcional & Descripción \\ \hline
				container\_id & string & No & Identificador del contenedor. \\ \hline
				operation & string & No & Nombre de la operación que se quiere ejecutar. Valores posibles: \textit{kill}, \textit{logs}, \textit{pause}, \textit{reload}, \textit{rename}, \textit{restart}, \textit{stop}, \textit{unpause}. \\ \hline
				data & dict & No & Argumentos de la operación. \\ \hline
			\end{tabular}
\end{adjustbox}
		\end{table}
	
		Tipos de operaciones:
		
		\subsubsection{rename}
			\begin{table}[h!]
				\centering
	\begin{adjustbox}{max width=\textwidth}
				\begin{tabular}{|l|l|l|l|l|}
					\hline
					Parámetro & Key & Tipo & Opcional & Descripción \\ \hline
					data &  & dict & No & Argumentos de la operación. \\ \hline
					& name & string & No & Nuevo nombre del contenedor. \\ \hline
				\end{tabular}
\end{adjustbox}
			\end{table}
		
		\subsubsection{logs}
			Respuesta:
			
			\begin{table}[h!]
				\centering
	\begin{adjustbox}{max width=\textwidth}
				\begin{tabular}{|l|l|l|}
					\hline
					Parámetro & Tipo & Descripción \\ \hline
					data & string & Logs del contenedor. \\ \hline
				\end{tabular}
\end{adjustbox}
			\end{table}
		
	
	
	
	
	\subsection{\textit{POST /deploy/image}}
		Permite ejecutar operaciones básicas en todos las imágenes.
	
		Cabeceras necesarias:
		\begin{table}[h!]
			\centering
	\begin{adjustbox}{max width=\textwidth}
			\begin{tabular}{|l|l|l|}
				\hline
				Nombre & Opcional & Descripción \\ \hline
				x-access-token & No & Token de acceso. \\ \hline
			\end{tabular}
\end{adjustbox}
		\end{table}
		
		Cuerpo de la petición (\textit{JSON}):
		
		\begin{table}[h!]
			\centering
	\begin{adjustbox}{max width=\textwidth}
			\begin{tabular}{|l|l|l|l|}
				\hline
				Parámetro & Tipo & Opcional & Descripción \\ \hline
				operation & string & No & Nombre de la operación que se quiere ejecutar. Valores posibles: \textit{list}, \textit{get}, \textit{prune}, \textit{pull}, \textit{remove}, \textit{search}. \\ \hline
				data & dict & No & Argumentos de la operación. \\ \hline
			\end{tabular}
\end{adjustbox}
		\end{table}
	
	\pagebreak
		Tipos de operaciones:
		
			\subsubsection{list}
				\begin{table}[h!]
					\centering
	\begin{adjustbox}{max width=\textwidth}
					\begin{tabular}{|l|l|l|l|l|}
						\hline
						Parámetro & Key & Tipo & Opcional & Descripción \\ \hline
						data &  & dict & No & Argumentos de la operación. \\ \hline
						& name & string & Sí & Mostrar sólo las imágenes pertenecientes a este repositorio. \\ \hline
						& all & bool & Sí & Mostrar todas las imágenes o no (incluídas las imágenes de capas intermedias). \\ \hline
						& filters & \hyperref[sec:filtroimagen]{Filtro} & Sí & Filtros adicionales. \\ \hline
					\end{tabular}
\end{adjustbox}
				\end{table}
			
				Respuesta (una sola imagen):
				\begin{table}[h!]
					\centering
	\begin{adjustbox}{max width=\textwidth}
					\begin{tabular}{|l|l|l|}
						\hline
						Parámetro & Tipo & Descripción \\ \hline
						data & \hyperref[sec:imagen]{Imagen} & Diccionario con los datos de la imagen. \\ \hline
					\end{tabular}
\end{adjustbox}
				\end{table}
			
				Respuesta (más de una imagen):
				\begin{table}[h!]
					\centering
	\begin{adjustbox}{max width=\textwidth}
					\begin{tabular}{|l|l|l|}
						\hline
						Parámetro & Tipo & Descripción \\ \hline
						total & int & Número de imágenes listadas. \\ \hline
						items & list[\hyperref[sec:imagen]{Imagen}] & Imágenes listadas. \\ \hline
					\end{tabular}
\end{adjustbox}
				\end{table}
			
			\subsubsection{get}
				\begin{table}[h!]
					\centering
	\begin{adjustbox}{max width=\textwidth}
					\begin{tabular}{|l|l|l|l|l|}
						\hline
						Parámetro & Key & Tipo & Opcional & Descripción \\ \hline
						data &  & dict & No & Argumentos de la operación. \\ \hline
						& name & string & No & Nombre de la imagen. \\ \hline
					\end{tabular}
\end{adjustbox}
				\end{table}
			
				Respuesta: \hyperref[sec:imagen]{Imagen}
				
			\subsubsection{prune}
				\begin{table}[h!]
					\centering
	\begin{adjustbox}{max width=\textwidth}
					\begin{tabular}{|l|l|l|l|l|}
						\hline
						Parámetro & Key & Tipo & Opcional & Descripción \\ \hline
						data &  & dict & No & Argumentos de la operación. \\ \hline
						& filters & \hyperref[sec:filtroimagen]{Filtro} & Sí & Filtros adicionales. \\ \hline
					\end{tabular}
\end{adjustbox}
				\end{table}
			
			\pagebreak
			\subsubsection{pull}
				\begin{table}[h!]
					\centering
	\begin{adjustbox}{max width=\textwidth}
					\begin{tabular}{|l|l|l|l|l|}
						\hline
						Parámetro & Key & Tipo & Opcional & Descripción \\ \hline
						data &  & dict & No & Argumentos de la operación. \\ \hline
						& repository & string & Sí & Repositorio e imagen a descargar. \\ \hline
						& tag & string & Sí & Tag de la imagen. \\ \hline
						& auth\_config & dict & Sí & Sobreescribir las credenciales. \\ \hline
						& platform & string & Sí & Plataforma en formato: os[/arch[/variant]] \\ \hline
					\end{tabular}
\end{adjustbox}
				\end{table}
			
				Respuesta: \hyperref[sec:imagen]{Imagen} descargada.
				
			\subsubsection{remove}
				\begin{table}[h!]
					\centering
	\begin{adjustbox}{max width=\textwidth}
					\begin{tabular}{|l|l|l|l|l|}
						\hline
						Parámetro & Key & Tipo & Opcional & Descripción \\ \hline
						data &  & dict & No & Argumentos de la operación. \\ \hline
						& image & string & No & Imagen a eliminar. \\ \hline
						& force & bool & Sí & Forzar borrado. \\ \hline
						& noprune & bool & Sí & Borrar o no imágenes padre sin tag. \\ \hline
					\end{tabular}
\end{adjustbox}
				\end{table}
			
			\subsubsection{search}
				\begin{table}[h!]
					\centering
	\begin{adjustbox}{max width=\textwidth}
					\begin{tabular}{|l|l|l|l|l|}
						\hline
						Parámetro & Key & Tipo & Opcional & Descripción \\ \hline
						data &  & dict & No & Argumentos de la operación. \\ \hline
						& term & string & No & Término de búsqueda. \\ \hline
					\end{tabular}
\end{adjustbox}
				\end{table}
			
				Respuesta:
				
				\begin{table}[h!]
					\centering
	\begin{adjustbox}{max width=\textwidth}
					\begin{tabular}{|l|l|l|l|}
						\hline
						Parámetro & Key & Tipo & Descripción \\ \hline
						total &  & int & Número de imágenes encontradas. \\ \hline
						items &  & list[dict] & Imágenes encontradas. \\ \hline
						& star\_count & int & Número de estrellas en DockerHub. \\ \hline
						& is\_official & bool & Si la imagen es oficial o no. \\ \hline
						& name & string & Nombre de la imagen \\ \hline
						& is\_automated & bool & Imagen automatizada. \\ \hline
						& description & string & Descripción. \\ \hline
					\end{tabular}
\end{adjustbox}
				\end{table}
		
	
	
	\pagebreak
	\subsection{\textit{POST /deploy/image/single}}
		Permite ejecutar operaciones básicas en una imagen en concreto.
		
		Cabeceras necesarias:
		\begin{table}[h!]
			\centering
	\begin{adjustbox}{max width=\textwidth}
			\begin{tabular}{|l|l|l|}
				\hline
				Nombre & Opcional & Descripción \\ \hline
				x-access-token & No & Token de acceso. \\ \hline
			\end{tabular}
\end{adjustbox}
		\end{table}
		
		Cuerpo de la petición (\textit{JSON}):
		
		\begin{table}[h!]
			\centering
	\begin{adjustbox}{max width=\textwidth}
			\begin{tabular}{|l|l|l|l|}
				\hline
				Parámetro & Tipo & Opcional & Descripción \\ \hline
				name & string & No & Nombre de la imagen. \\ \hline
				operation & string & No & Nombre de la operación que se quiere ejecutar. Valores posibles: \textit{history}, \textit{reload}. \\ \hline
				data & dict & No & Argumentos de la operación. \\ \hline
			\end{tabular}
\end{adjustbox}
		\end{table}
	
		Tipos de operaciones:
		
			\subsubsection{history}
				Respuesta:
				
				\begin{table}[h!]
					\centering
	\begin{adjustbox}{max width=\textwidth}
					\begin{tabular}{|l|l|l|}
						\hline
						Parámetro & Tipo & Descripción \\ \hline
						data & string & Historia de la imagen. \\ \hline
					\end{tabular}
\end{adjustbox}
				\end{table}
			
			\subsubsection{reload}
				Recarga la imagen y aplica los posibles cambios que tenga.
		
		
	



