\chapter{Instalación del sistema}

Una vez clonado el repositorio puedes instalar y ejecutar tanto backend como frontend siguiendo los pasos que se describen a continuación.

\section{Backend}
El backend de IPManager tiene algunas dependencias que tienes que instalar para que funcione correctamente, son las siguientes:

\begin{lstlisting}
apt install sshpass
apt instal openssl
apt install libffi6

pip3 install -r requirements.txt
\end{lstlisting}

Asegúrate que la versión instalada del paquete \textit{werkzeug} coincide con 0.16.1.

Para ejecutarlo puedes hacerlo mediante Flask o con Gunicorn.

\begin{itemize}
	\item Flask (no recomendado):
\begin{lstlisting}
export FLASK_APP=wsgi.py

flask run
\end{lstlisting}
	
	\item Gunicorn:
\begin{lstlisting}
gunicorn -b 0.0.0.0:5000 wsgi:app
\end{lstlisting}
	
\end{itemize}

\subsection{Docker}
El backend también está disponible en Docker, puedes descargarte la imagen de la siguiente manera:

\begin{lstlisting}
docker pull harvestcore/ipm-backend:<tag>
\end{lstlisting}

Se recomienda siempre utilizar la última versión disponible de la imagen, la cual puede consultarse \href{https://github.com/harvestcore/tfg/releases}{aquí} o \href{https://hub.docker.com/r/harvestcore/ipm-backend/tags}{aquí}. Revisa también las [variables de entorno](./env-vars.md) necesarias para ejecutar el backend.

Ejemplo de ejecución:

\begin{lstlisting}
docker run -e MONGO_HOSTNAME=172.20.0.2 harvestcore/ipm-backend:<tag>
\end{lstlisting}

Si por el contrario quieres construir tú mismo la imagen, ejectuta:

\begin{lstlisting}
cd backend

docker build . -t ipm-backend:<tag>
\end{lstlisting}





\section{Frontend}

Para instalar el frontend primero revisa y configura las [variables de entorno](./env-vars.md), tras eso solo tienes que ejecutar lo siguiente:

\begin{lstlisting}
cd frontend

// Construir el frontend
npm build --prod
\end{lstlisting}

Para ejecutarlo se recomienda utilizar Nginx u otro tipo de servidor web. En la raíz del frontend se adjunta el archivo de configuración (\textit{nginx.conf}) usado para construir la imagen de Docker, y tambien puede ser usado en este caso.



\subsection{Docker}

Puedes ejecutar el frontend con Docker, para ello puedes bajarte la imagen del repositorio disponible o puedes construir y ejecutar tú mismo la imagen.

\begin{lstlisting}
docker pull harvestcore/ipm-frontend:<tag>
\end{lstlisting}

El tag o versión lo puedes consultar \href{https://github.com/harvestcore/tfg/releases}{aquí} o \href{https://hub.docker.com/r/harvestcore/ipm-frontend/tags}{aquí}. Se recomienda usar siempre la última versión estable.

\begin{lstlisting}
cd frontend

// Construir imagen
docker build . -t ipm-frontend:<tag>

docker run ipm-frontend:<tag>
\end{lstlisting}



\section{Docker-compose}

En el caso de utilizar el docker-compose que se encuentra en la raíz del repositorio solo es necesario ejecutar lo siguiente:

\begin{lstlisting}
docker-compose build

docker-compose up
\end{lstlisting}

Por supuesto se pueden agregar variables de entorno para configurar el backend. Un ejemplo sería:

\begin{lstlisting}
docker-compose up -e BASE_DATABASE=ipm_root
\end{lstlisting}

El docker-compose tiene configurada una red bridge con la siguiente subnet:

\begin{itemize}
	\item 172.20.0.0/16
\end{itemize}

Por otro lado las máquinas cuentan con las siguientes direcciones IP estáticas asignadas:

\begin{itemize}
	\item mongo: 172.20.0.2
	\item ipmanager-backend: 172.20.0.3
	\item ipmanager-frontend: 172.20.0.4
\end{itemize}

También se fija la variable de entorno \textit{BASE\_DATABASE} con valor \textit{ipm\_root}.