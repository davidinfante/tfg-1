\chapter{Instalación del sistema}

Una vez clonado el repositorio se puede instalar y ejecutar tanto backend como frontend siguiendo los pasos que se describen a continuación.

\section{Backend}
El backend de \textbf{IPManager} tiene algunas dependencias que se tienen que instalar para que funcione correctamente, son las siguientes:

\begin{lstlisting}
apt install sshpass
apt instal openssl
apt install libffi6

pip3 install -r requirements.txt
\end{lstlisting}

La versión instalada del paquete \textit{werkzeug} debe ser 0.16.1.

Se puede ejecutar con \textit{Flask} o \textit{Gunicorn}.
\begin{itemize}
	\item Flask:
\begin{lstlisting}
export FLASK_APP=wsgi.py

flask run
\end{lstlisting}
	
	\item Gunicorn:
\begin{lstlisting}
gunicorn -b 0.0.0.0:5000 wsgi:app
\end{lstlisting}
	
\end{itemize}

\subsection{\textit{Docker}}
El backend también está disponible en \textit{Docker}, la imagen puede descargarse de la siguiente manera:

\begin{lstlisting}
docker pull harvestcore/ipm-backend:<tag>
\end{lstlisting}

Se recomienda siempre utilizar la última versión disponible de la imagen, la cual puede consultarse en la sección \textit{Releases} del repositorio en  \href{https://github.com/harvestcore/tfg/releases}{\textit{GitHub}} \cite{ipmanagerreleases} o en la sección \textit{tags} del repositorio en \href{https://hub.docker.com/r/harvestcore/ipm-backend/tags}{\textit{DockerHub}} \cite{dockerhubtagback}. Deben comprobarse también las \hyperref[sec:variables]{variables de entorno} necesarias para ejecutar el backend.

Ejemplo de ejecución:

\begin{lstlisting}
docker run -e MONGO_HOSTNAME=172.20.0.2 harvestcore/ipm-backend:<tag>
\end{lstlisting}

En el caso de querer construir la imagen se debe ejecutar:

\begin{lstlisting}
cd backend

docker build . -t ipm-backend:<tag>
\end{lstlisting}





\section{Frontend}

Para instalar el frontend se deben revisar y configurar las \hyperref[sec:variables]{variables de entorno}, tras eso se debe ejecutar lo siguiente:

\begin{lstlisting}
cd frontend

npm build --prod
\end{lstlisting}

Para ejecutarlo se recomienda utilizar \textit{Nginx} u otro tipo de servidor web. En la raíz del frontend se adjunta el archivo de configuración (\textit{nginx.conf}) usado para construir la imagen de \textit{Docker}, y tambien puede ser usado en este caso.



\subsection{\textit{Docker}}

Se puede ejecutar el frontend con \textit{Docker}, para ello se puede descargar la imagen del repositorio disponible o se puede construir de forma local.

\begin{lstlisting}
docker pull harvestcore/ipm-frontend:<tag>
\end{lstlisting}

El tag o versión se puede consultar en la sección \textit{Releases} del repositorio en  \href{https://github.com/harvestcore/tfg/releases}{\textit{GitHub}} \cite{ipmanagerreleases} o en la sección \textit{tags} del repositorio en \href{https://hub.docker.com/r/harvestcore/ipm-frontend/tags}{\textit{DockerHub}} \cite{dockerhubtagfront}. Se recomienda usar siempre la última versión estable.

\begin{lstlisting}
cd frontend

// Construir imagen
docker build . -t ipm-frontend:<tag>

docker run ipm-frontend:<tag>
\end{lstlisting}



\section{Docker-compose}

En el caso de utilizar el \textit{docker-compose} que se encuentra en la raíz del repositorio solo es necesario ejecutar lo siguiente:

\begin{lstlisting}
docker-compose build

docker-compose up
\end{lstlisting}

Por supuesto se pueden agregar variables de entorno para configurar el backend. Un ejemplo sería:

\begin{lstlisting}
docker-compose up -e BASE_DATABASE=ipm_root
\end{lstlisting}

El docker-compose tiene configurada una red \textit{bridge} con la siguiente \textit{subnet}:

\begin{itemize}
	\item 172.20.0.0/16
\end{itemize}

Por otro lado las máquinas cuentan con las siguientes direcciones IP estáticas asignadas:

\begin{itemize}
	\item mongo: 172.20.0.2
	\item ipmanager-backend: 172.20.0.3
	\item ipmanager-frontend: 172.20.0.4
\end{itemize}

También se fija la variable de entorno \textit{BASE\_DATABASE} con valor \textit{ipm\_root}.