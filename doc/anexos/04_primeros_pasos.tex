\chapter{Primeros pasos y \textit{CLI}}

\section{Inicialización de la base de datos}
Tras instalar el backend debes ejecutar un script llamado \textsf{init\_database.py} (se encuentra en la raíz del proyecto), el cual creará un usuario en el cliente base. Este usuario es administrador y sus credenciales deben ser cambiadas una vez haya sido creado.

El cliente base viene denominado por el nombre de la base de datos principal, el cual se toma de la variable de entorno \textsf{BASE\_DATABASE} (toma valor \textsf{ipm\_root} en caso de no encontrarse configurada).

Tras crear este primer usuario administrador, este puede comenzar a crear otros usuarios o clientes usando la \textit{API} o el \textit{CLI}.

\section{\textit{CLI}}

El \textit{CLI} que se incluye en el directorio raíz del backend permite realizar algunas operaciones con clientes (o \textit{customers}) y usuarios. Se puede ejecutar de la siguiente manera:

\begin{lstlisting}
cd backend

python3 cli.py
\end{lstlisting}

Es interactivo y permite:

\begin{itemize}
	\item Crear un cliente.
	\item Activar un cliente.
	\item Desactivar un cliente.
	\item Agregar un usuario a un cliente.
\end{itemize}

Internamente utiliza la configuración de las [variables de entorno](env-vars.md) que se encuentre establecida en el momento de utilizar el \textit{CLI}.