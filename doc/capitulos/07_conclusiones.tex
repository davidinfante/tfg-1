\chapter{Conclusiones y trabajos futuros}

Tras el desarrollo de este proyecto se puede comprobar que los objetivos que se marcaron al inicio se han alcanzado exitosamente.

\textbf{IPManager} es una solución que responde al problema que se plantea ya que unifica los procesos de aprovisionado de sistemas, despliegue de servicios y almacenamiento de configuraciones de una manera sencilla y modular. Además, basada en una arquitectura de microservicios, provee de un backend y un frontend que pueden funcionar de forma independiente el uno del otro.

Es también una solución liviana y no intrusiva ya que se puede instalar en sistemas con pocos recursos de manera local o a través de contenedores \textit{Docker}, o incluso se puede desplegar en sistemas \textit{Cloud}.

Ofrece tambien una \textit{API REST} que posibilita las comunicaciones con otros sistemas. Un ejemplo de este uso es \textbf{IPMDroid}, una aplicación móvil para Android que se ha desarrollado de forma paralela a este proyecto para la asignatura ''Programación de Dispositivos Móviles''. Esta ofrece una interfaz sencilla para la administración de configuraciones y despliegue de servicios y se puede encontrar en \href{https://github.com/harvestcore/ipmdroid}{este repositorio} \cite{ipmdroid} en \textit{Github}.

Además, todo el proyecto está compuesto por software libre bajo la licencia \href{https://www.gnu.org/licenses/gpl-3.0.html}{GNU GPLv3} y se encuentra en \href{https://github.com/harvestcore/tfg}{este repositorio} \cite{ipmanager}.

Por último, y no por ello menos importante, se ha aprendido a desarrollar un proyecto de grandes dimensiones desde las etapas más tempranas, como son las primeras investigaciones e ideas, hasta las finales, dejando la puerta abierta a posibles mejoras y actualizaciones.


\pagebreak
En cuanto a trabajos futuros hay ciertos aspectos que se podrían mejorar:

\begin{itemize}
	\item Actualmente el módulo para despliegues de servicios solo permite realizar operaciones con imágenes y contenedores, por lo que una mejora sería integrar el manejo de redes, volúmenes u otras configuraciones relacionadas con \textit{Docker}.
	\item Se podrían ampliar los datos que se almacenan en el módulo de configuraciones. Las actuales se centran en configuraciones de red, pero podrían ampliarse para almacenar cualquier otro aspecto.
	\item El módulo de aprovisionamiento se centra exclusivamente en la ejecución de \textit{Playbooks}. Una mejora en este aspecto podría ser un mejor manejo de los parámetros de ejecución de estos.
	\item El frontend desarrollado intenta ser sencillo a la par que funcional, pero hay características que se podrían mejorar. Una mejora podría ser las vistas individuales de las configuraciones de las máquinas que se almacenan.
\end{itemize}

