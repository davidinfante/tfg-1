\chapter{Estado del arte}

En este capítulo repasan de las diferentes soluciones que existen actualmente al problema expuesto. Cada una de ellas destaca sobre las demás en un aspecto u otro, pero ninguna llega a aunar todas las características deseadas. Tras eso se hace una crítica a las soluciones vistas y se comenta el por qué no son ideales.

\section{Soluciones actuales}

En el ámbito de los despliegues y automatización hay una gran cantidad de software que en mayor o menor medida permiten realizar estas tareas. Con características similares a las de este proyecto destacan tres y son las siguientes:

\begin{itemize}
	\item \textbf{GECOS}. \textit{Guadalinex Escritorio Corporativo Estándar} es un proyecto de la Junta de Andalucía que ofrece una distribución de \textit{GNU Linux} centrada en la administración de sistemas. Este ecosistema, que está basado totalmente en software libre, actúa como centro de control, que permite el despliegue de equipos, su soporte y administración. Además cuenta con un repositorio de software para proveer a las máquinas que se aprovisionan. Utiliza herramientas y tecnologías como \textit{MongoDB} para la gestión de los datos, \textit{Chef} para el aprovisionamiento o \textit{Celery} para el manejo de las colas de tareas.
	
	Aunque no se trata de un sistema altamente intrusivo para las máquinas que gestiona, este sí requiere que se instale una imagen en la máquina que se vaya a utilizar como maestra. De este modo sólo un equipo es el que tiene acceso a la administración del resto. La gestión de los sistemas se hace desde una interfaz web y permite organizar los equipos según los criterios que se desee e identifica a cada uno de ellos mediante certificados digitales únicos.
		
	\item \textbf{Ansible Tower}. Es una solución de \textit{Red Hat} para la administración de ecosistemas de ámbito empresarial ya que se centra en multitud de sistemas operativos, servidores, infraestructuras virtuales y redes. Esta gestión se realiza desde una interfaz web o se puede integrar con cualquier otro sistema mediante una \textit{API REST}.
	
	La principal característica es que todos los procesos se pueden realizar de forma muy visual y con poca configuración por parte del administrador. Se centra principalmente en el aprovisionado de sistemas mediante \textit{Ansible}, aunque también permite monitorizar estas máquinas y gestionar permisos a los diferentes usuarios.
		
	\item \textbf{Portainer}. Se trata de una herramienta de gestión y administración de contenedores que amplía las funcionalidades que ofrece \textit{Kinematic}, la \textit{GUI} oficial a la que \textit{Docker} da soporte. Al igual que las anteriores soluciones \textit{Portainer} ofrece sus capacidades a través de una interfaz web y es la más liviana de todas ya que se puede ejecutar fácilmente en un contenedor.
	
	Permite administrar todos los aspectos de \textit{Docker} como son las imágenes, contenedores, redes y volúmenes entre otros. Además la interacción con todos estos aspectos es muy sencilla. Se trata también de un proyecto \textit{open source}, por lo que es gratuíto.
\end{itemize}


\section{Crítica}

Si bien todas estas herramientas mencionadas permiten realizar las funcionalidades que se desean ninguna de ellas las aúna todas. En primer lugar \textit{GECOS} ofrece una herramienta de administración de sistemas y de aprovisionado, pero no ofrece ningún tipo de despliegue de servicios. \textit{Red Hat} hace uso de su herramienta estrella en el aprovisionado de sistemas y la lleva al siguiente nivel con \textit{Ansible Tower}, pero al igual que con \textit{GECOS} no ofrece nada relevante en cuando a servicios en contenedores. Por último \textit{Portainer} provee de una excelente solución para este problema, pero no para el aprovisionado y gestión de configuraciones.

Otro aspecto importante es el económico. \textit{GECOS} y \textit{Portainer} son sistemas completamente gratuítos, por lo que serían una opción ideal. En el caso del segundo, \textit{Portainer} ofrece planes superiores con mayor soporte y funcionalidades avanzadas, pero la versión principal es gratuíta. En cambio \textit{Ansible Tower} es de pago, teniendo la suscripción más básica un coste aproximado anual de 5000 dólares anuales y la más avanzada de unos 14000 dólares. Si bien estos precios pueden ser asequibles para una gran organización, para una pequeña o mediana empresa pueden ser inasumibles.