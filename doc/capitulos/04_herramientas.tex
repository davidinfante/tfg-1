\chapter{Herramientas y tecnologías utilizadas}

Este capítulo desarrolla en profundidad todas las decisiones tomadas respeto a las herramientas y tecnologías que se han utilizado en \textbf{IPManager}. En cada una de las secciones se hace un breve repaso de las diferentes soluciones que existen actualmente y se exponen los motivos por los que se sigue un rumbo u otro.

Se comienza por el aspecto más básico, la arquitectura de microservicios seleccionada, y se continúa con la integración continua, \textit{GitHub Actions}; y el despliegue en contenedores, \textit{Docker}. Tras esto se elige la herramienta a utilizar en al aprovisionado de sistemas, \textit{Ansible}; y el sistema de gestión de base de datos, \textit{MongoDB}. Finalmente se desarrollan las decisiones tomadas en cuanto a \textit{frameworks} y lenguajes de programación usados tanto en el backend, \textit{Python} y \textit{Flask}, como en el frontend, \textit{Angular}.

\section{Arquitectura}

En este aspecto tan esencial la elección está bastante clara y opto por la arquitectura de microservicios. En el proyecto que se desarrolla no tiene sentido un sistema monolítico en el que frontend y backend se encuentren juntos. En este caso las funcionalidades a desarrollar se centran en aspectos muy concretos y los beneficios de este tipo de arquitectura superan a los inconvenientes.

Entre estos beneficios descato en primer lugar la modularidad, ya que el backend sería totalmente independiente del frontend, lo que permite que se puedan desarrollar infinitas interfaces de usuario, cada una adecuada al uso que se vaya a dar del sistema. Por otro lado destaco la escalabilidad, ya que en el caso de necesitar más instancias se pueden desplegar de forma rápida y sencilla; y finalmente la seguridad y aislamiento, ya que cada instancia del sistema no interviene con las demás, aislando en este caso los datos de cada caso de uso.

Tradicionalmente el software ha tenido una arquitectura monolítica en la que todos los servicios y funcionalidades están integrados y programados en un mismo sistema, pero como se indica antes, no tiene sentido en este tipo de proyecto.

Las principales ventajas de los microservicios sobre la arquitectura monolítica son:
\begin{itemize}
	\item Agilidad. No es necesario desarrollar todas las funcionalidades completas, por lo que se pueden reutilizar otros microservicios ya desarrollados para suplir las necesidades actuales.
	\item Modularidad. Cada microservicio es independiente del resto, lo que facilita el desarrollo y el despliegue de estos.
	\item Escalabilidad. Debido a la modularidad de estos la escalabilidad horizontal es asequible y muy beneficiosa.
	\item Seguridad y aislamiento. Cada uno de estos servicios encapsula toda su funcionalidad, quedando aislados del resto. Cualquier tipo de vulnerabilidad de la seguridad queda reducido a una parte del sistema general, evitando así pérdidas de información y posibles fallos a otros microservicios.
\end{itemize}

En cambio este sistema también tiene desventajas. Tener multitud de servicios en ejecución conlleva una configuración y un coste de implantación más alto de lo habitual, además de que no hay una uniformidad a la hora de desempeñar un despliegue. Esto conlleva una complejidad añadida ya que aunque los servicios son mas ligeros y sencillos el sistema que conforman es mucho más complejo. La administración también es más compleja ya que se requieren conocimientos específicos de cada microservicio para este cometido.


\section{Integración continua}

La integración continua es una práctica que consiste en el control de versiones del código que se desarrolla y en la ejecución de pruebas automáticas del mismo de forma periódica con el fin de detectar errores o un mal funcionamiento de una forma rápida. Actualmente es un requisito necesario e indispensable en cualquier tipo de software y debe abordar todos los aspectos del mismo.

\textit{GitHub Actions} es el sistema elegido para la integración continua. Esto se debe a dos motivos, el primero es que el proyecto se encuentra alojado en \textit{GitHub} y por otro lado \textit{GitHub Actions} permite la ejecución de contenedores \textit{Docker}. Este último aspecto es muy importante en el software que se desarrolla debido a que el backend trabaja con este tipo de tecnología y los tests unitarios deben probar todas estas funcionalidades. Además para las pruebas del resto de funcionalidades el poder ejecutar un contenedor con una base de datos facilita mucho el proceso de integración continua. Así mismo son sistemas que se encuentran perfectamente integrados el uno con el otro.

\bigskip
Existen multitud de servicios para este tipo de pruebas:
\begin{itemize}
	\item Jenkins
	\item Travis CI
	\item Bamboo
	\item GitHub Actions
	\item GitLab CI
	\item Circle CI
\end{itemize}

Todos ellos comparten características y son realmente similares. Algunos como \textit{Jenkins} permiten la integración de multitud de plugins y otros permiten además despliegues continuos. En el caso de este software la característica que más nos interesa es que sea gratuito y la mayoria de ellos lo son para proyectos de software libre. Esta es otra de las características que hacen que estos sistemas gratuitos tengan tanta popularidad.




\section{Despliegue del software en contenedores}

En este ámbito se va a utilizar \textit{Docker} para desplegar tanto frontend como backend. Este sistema provee una capa adicional de abstracción y de virtualización de las aplicaciones, lo que nos permite ejecutar un software de manera aislada sin tener que depender de complejas configuraciones de máquinas virtuales o hipervisores. Por otro lado los recursos pueden ser también aislados.

Para la creación de estos contenedores se utilizan los denominados \textit{Dockerfile}, que son archivos de texto plano con las diferentes instrucciones que crearán a voluntad el entorno de ejecución de nuestro software. En el caso de este proyecto se van a crear dos de estos archivos, uno el frontend y otro para el backend, y la creación y despliegue de las imágenes generadas se realizará mediante \textit{DockerHub}.

Como se ha indicado anteriormente este proceso también se puede automatizar, ya sea con scripts creados a mano o con el uso los hooks de \textit{DockerHub}, que construyen las imágenes específicas cada vez que se haga un cambio en el código o cada vez que se produzca un evento concreto.

\textit{Docker Compose} es otra herramienta que facilita aún más este proceso, ya que a partir de un archivo \textit{YAML} permite crear estos contenedores, configurarlos y conectarlos de una forma muy sencilla. En este proyecto también se incluye uno de estos archivos.

Finalmente existen otras herramientas muy interesantes, como son \textit{Kubernetes}, \textit{OpenShift} o \textit{Mesos}, que nos permiten orquestar y escalar los contenedores según los criterios que se configuren. En este proyecto no se hace uso de ellas pero en el caso de que se quisiera dar un paso más en el despliegue del software podrían ser muy interesantes.


\section{Aprovisionamiento}

Existen muchas herramientas que permiten el aprovisionamiento de sistemas, algunas más centradas en el ámbito \textit{Cloud} y otras de uso local.

En el caso de este proyecto la opción que mejor encaja es \textit{Ansible}. Debido a su sencillez de uso y de no requerir un servidor central lo hace ideal para el uso en el backend. Además, ya que se encuentra desarrollado en \textit{Python} y a que existe un \textit{SDK}, la integración es inmediata, solo teniendo que desarrollar las funcionalidades que se quieran.

Por otro lado la sencillez de los \textit{Playbooks} lo hace más atractivo aún, ya que la sintaxis de los archivos \textit{YAML}  es muy sencilla. En cuanto a la conexión mediante \textit{SSH} solo se requiere que el backend tenga conexión a internet, por lo que no es necesario ningún protocolo o configuración adicional para que funcione.

\bigskip
Algunas otras alternativas son:
\begin{itemize}
	\item Chef. La configuración de las máquinas se hace de forma procedural y se depende de un servidor central que almacene las configuraciones o \textit{recetas}. Además ofrece análisis e informes de las máquinas aprovisionadas.
	\item Puppet. Es un conjunto de herramientas que permiten orquestar y administrar grandes conjuntos de máquinas. Al igual que \textit{Chef} depende de un servidor central y permite ampliar su funcionalidad a través de módulos.
\end{itemize}

\section{Base de datos}


Elegir un sistema de gestión de base de datos es una de las decisiones más importantes a la hora de diseñar y desarrollar un software. Existe una gran variedad de tipos de bases de datos y hay que tener en cuenta una serie de cuestiones que serán determinantes a la hora de elegir un tipo u otro. En este proyecto se va a usar una base de datos \textit{NoSQL}, concretamente \textit{MongoDB}.

Las bases de datos de tipo \textit{SQL} se basan en las relaciones entre los datos. Estos se introducen en registros y luego se organizan por tablas, columnas y tuplas, permitiendo relacionarlos de manera sencilla. El principal lenguaje de consultas es el \textit{Standard Query Language} (\textit{SQL}), el cual esta compuesto por una serie de comandos de diferentes tipos, que se usan para unos cometidos u otros. Sus principales características son el esquema rígido que se define previo al uso que garantiza el esquema \textit{ACID}.

Por las características del proyecto este modelo queda excluido, ya que el tipo de datos que se va a manejar no requiere de grandes relaciones entre ellos y además el esquema puede ser cambiante. Podría ocurrir que ciertos valores no se encontraran almacenados, bien porque no son necesarios o bien porque el usuario decide no insertarlos, por lo que sería mantener una estructura que no se está cumpliendo.

Por otro lado el tipo de consultas que se van a realizar no son extremadamente complejas. Los datos manejados no tienen relaciones entre sí y las consultas serían realmente básicas. Otro punto a favor en este aspecto para las bases de datos \textit{NoSQL} es la velocidad a la hora de realizar las consultas.

En el caso de la integración con el software en las bases de datos \textit{SQL} se utilizan los llamados \textit{ORM} (\textit{Object Relation Mapper}). Estos permiten realizar consultas a estas bases de datos de una forma más amigable en el lenguaje que se esté usando, lo que implica que se tenga que volver a redefinir el esquema para poder manejar estos datos. En cambio con \textit{NoSQL} esta integración suele ser mas sencilla, al utilizarse directamente objetos como diccionarios.

Dentro de las bases de datos \textit{NoSQL} existen diferentes tipos. En el caso de este proyecto el modelo que mas encaja es el documental, en la que una semiestructura flexible almacenada en forma de documentos es ideal. \textit{MongoDB} es una gran elección, ya que los datos se almacenan en \textit{BSON} (\textit{Binary JSON}), lo que ofrece aún más flexibilidad a la hora de almacenar objetos. También permite crear índices en cualquier clave y el balanceo de carga en el caso de realizar grandes cantidades de consultas simultáneas.

Actualmente también están destacando las bases de datos en la nube o \textit{DBaaS} (\textit{DataBase as a Service}), las cuales estan optimizadas para operaciones en entornos virtualizados. La principal característica de estos servicios es que se suele pagar por el uso de almacenamiento y además conceptos como la escalabilidad o la alta disponibilidad estan asegurados. En el caso de \textit{MongoDB} existe \textit{Atlas}, que incluso ofrece planes gratuitos. Otro ejemplo de \textit{DBaaS} sería \textit{mLab}, muy similar al anterior pero con una configuración más sencilla. Para el desarrollo de este software este aspecto es muy interesante, ya que al tratarse de un microservicio el no estar atado a una base de datos local permite que se pueda desplegar tambien en la nube.



\section{Backend}

\subsection{Lenguaje de programación y framework}

Actualmente existen multitud de lenguajes de programación que podrían usarse sin problema alguno para desarrollar una API de las características que se requieren. Las características deseadas que deberían ofrecernos estos en el caso de este software son sencillas aunque a la par difíciles de encontrar en ocasiones. Algunos de estos lenguajes que se suelen utilizar en el desarrollo de \textit{APIs} son \textit{Java}, \textit{JavaScript}, \textit{PHP}, \textit{Python}, \textit{Ruby}, \textit{C\#} y \textit{Go}, entre otros.

\begin{itemize}
	\item Para el manejo de datos se requiere que se pueda conectar a \textit{MongoDB} y esto lo cumplen los lenguajes mencionados, por lo que todos son buenos candidatos en este caso.
	\item En cuanto al aprovisionamiento se requiere algún tipo de \textit{SDK} de \textit{Ansible}. Los lenguajes que soportan esto son \textit{Go}, \textit{Python}, \textit{Ruby}, \textit{PHP} y \textit{JavaScript}, mientras que el resto tienen un soporte limitado.
	\item Se debe poder administrar \textit{Docker} y en este caso, al igual que con \textit{MongoDB}, todos los lenguajes tienen \textit{SDKs} disponibles.
\end{itemize}

Por otro lado, debido a la arquitectura del proyecto, no es necesario que el \textit{framework} elegido tenga la arquitectura \textit{MVC} ya que de la vista se encarga el frontend. Por este motivo todos aquellos que siguen este modelo quedan descartados. Podrían utilizarse sin problema alguno, pero no tendría mucho sentido ya que no le estaríamos sacando todo el partido posible a los mismos.

Anteriormente se mencionaba como aspecto importante los recursos que tienen estos lenguajes y cierto es que algunos pueden ofrecer más que otros, bien sea porque son más antiguos o bien porque son más usados y la comunidad es mayor. En esto destaca \textit{Python}, también motivado por opinión personal, ya que existen una infinidad de librerías y recursos para este lenguaje y se puede desarrollar cualquier aplicación de forma sencilla e intuitiva. Además, en el caso del aprovisionamiento, \textit{Ansible} está programado en \textit{Python}, por lo que la integración con su \textit{SDK} sería directa, sin problema alguno.

La ausencia de tipado en \textit{Python} da una mayor flexibilidad y libertad a la hora de desarrollar, pero puede inducir a errores, por lo que es necesario tener un especial cuidado. Otro aspecto es que se trata de un lenguaje interpretado, algo que agiliza el desarrollo ya que no hay que emplear tiempo extra en el compilado. También tiene una sintaxis muy sencilla que facilita la comprensión del código.

En cuanto al \textit{framework}, como se indicaba antes no es necesario que disponga de arquitectura \textit{MVC}, por lo que \textit{Django} queda descartado. En su defecto se usará \textit{Flask} junto a otros módulos como \textit{Flask-RESTPlus} o \textit{Marshmallow} (usado para la definición de esquemas).

La elección de un lenguaje u otro también depende de los recursos que nos ofrezca, esto es librerías, \textit{frameworks} y todas aquellas características que hagan destacar un lenguaje sobre otro. También influye la experiencia que se tenga, ya que afrontar un gran proyecto con un lenguaje que nunca has usado puede ser un gran reto.


\bigskip
Otros \textit{frameworks} que se han tenido en cuenta a la hora de esta elección han sido:

\begin{itemize}
	\item \textbf{Java}. Destacan \textit{frameworks} como \textit{Spring} y \textit{Struts}.	
	\item \textbf{JavaScript}. \textit{Express} es el más utilizado y se ejecuta sobre \textit{Node.js}. Otros ejemplos son \textit{Sails} o \textit{Meteor}.	
	\item \textbf{PHP}. Los más popular son \textit{Slim} y \textit{Lumen}. Ambos son \textit{microframeworks} bastante sencillos con muchas funcionalidades incorporadas.	
	\item \textbf{Ruby}. \textit{Roda} y \textit{Sinatra} son los más utilizados.	
	\item \textbf{C\#}. El más popular es \textit{.NET Core}, de \textit{Microsoft}.
	\item \textbf{Go}. Tanto \textit{Revel} como \textit{Gin} son los más usados.
\end{itemize}






\section{Frontend}

En el caso de este proyecto la primera elección que se toma es la de abandonar el \textit{stack} \textit{HTML}/\textit{CSS}/\textit{JS} ya que realmente no ofrece nada novedoso sobre las demás soluciones. Tras esto, elijo \textit{Angular}.

\textit{Angular} es un \textit{framework} que aborda todos los aspectos del desarrollo frontend, desde la parte visual hasta las comunicaciones. Su arquitectura se basa en componentes que se pueden crear y personalizar a voluntad y el lenguaje usado para su desarrollo es \textit{TypeScript}, lo cual permite un mayor control de los datos que se manejan. Integra además multitud de librerías, como \textit{RxJS}, para aprovechar sus virtudes. Entre sus principales características destacan el enlace de datos bidireccional (\textit{2-way data-binding}) entre el modelo y la vista, la inyección de servicios y dependencias, que facilita el desarrollo y la comprensión del código, y la validación de datos y mecanismos de seguridad integrados. Por contra la curva de aprendizaje es mas grande, ya que integra multitud de conceptos diferentes que no se contemplan en las demás soluciones.

\bigskip
Por otro lado otras opciones que se han barajado han sido:

\textbf{React}. Tiene un enfoque reactivo y trabaja con un \textit{DOM} virtual en varias capas, lo que permite que sólo se actualicen aquellas partes de la página que deban actualizarse. También permite la creación y reutilización de componentes personalizados, lo que dota a esta librería de mucha flexibilidad a la hora del desarrollo. Por contra sólo se trata de una librería centrada en la parte visual del frontend, por lo que el manejo de datos entre componentes o cualquier tipo de comunicación externa, como \textit{HTTP}, quedan a cargo del desarrollador.

\textbf{Vue}. En este caso Vue comparte conceptos de \textit{React} y de \textit{Angular}, por lo que sería el punto intermedio entre ambos. Destaca por ser mas liviano que estos y por su simplicidad a la hora del desarrollo, lo que hace que la curva de aprendizaje sea bastante menor. Por el contrario, al igual que con \textit{React}, las comunicaciones corren a cargo del programador, lo que es un punto en su contra.

\bigskip
En cuanto a estas alternativas, aunque es muy interesante el enfoque que tienen se quedan cortas a la hora de la comunicación con el backend. El desarrollador es el que debe proveer de los métodos de comunicación, lo que requiere más tiempo. En cambio con \textit{Angular} esos aspectos ya se encuentran integrados, lo que simplifica mucho el proceso. Por otro lado aspectos como el \textit{2-way data-binding} y la validación de datos son también interesantes, ya que aportan flexibilidad y agilizan el desarrollo.

Por estos motivos \textit{Angular} es la mejor solución en el caso de este software y en caso de quedarse corta en algunos aspectos, permite la integración de otras librerías. Además \textit{Angular} incluye librerías y mecanismos para tests unitarios e integración continua, algo muy necesario en el desarrollo de un software.

Finalmente para los test unitarios y \textit{end-to-end} (\textit{e2e}) se van a utilizar herramientas que se integran perfectamente con \textit{Angular}. Para los primeros se utiliza \textit{Karma}, que viene incluido por defecto en el \textit{framework} y permite comprobar el funcionamiento unitario de cada uno de los componentes.

Para los tests \textit{e2e} se utilizará \textit{Cypress}, una herramienta gratuita muy potente que permite realizar tests al frontend como si de un usuario se tratara, haciendo uso de datos sobre las diferentes funcionalidades del sistema y además permite ejecutar estos tests en casi todos los navegadores actuales. En este aspecto se han tenido en cuenta opciones como \textit{Puppeteer} y \textit{Nightwatch.js}, pero se descartaron pues la sintaxis era mas compleja.







